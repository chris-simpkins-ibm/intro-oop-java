\documentclass{beamer}

\newcommand{\lesson}{Iterators and Streams}


\newcommand{\course}{Introduction to Object-Oriented Programming}
\subject{\course}
\title[\lesson]{\course}
\subtitle{\lesson}

\author[CS 1331]
{Christopher Simpkins \\\texttt{chris.simpkins@gatech.edu}}
\institute[Georgia Tech]

\date[]{}

\newcommand{\link}[2]{\href{#1}{\textcolor{blue}{\underline{#2}}}}
\newcommand{\code}{http://www.cc.gatech.edu/~simpkins/teaching/gatech/cs1331/code}

\usepackage{colortbl}

% If you have a file called "university-logo-filename.xxx", where xxx
% is a graphic format that can be processed by latex or pdflatex,
% resp., then you can add a logo as follows:

% \pgfdeclareimage[width=0.6in]{coc-logo}{cc_2012_logo}
% \logo{\pgfuseimage{coc-logo}}

\mode<presentation>
{
  \usetheme{Berlin}
  \useoutertheme{infolines}

  % or ...

 \setbeamercovered{transparent}
  % or whatever (possibly just delete it)
}

\usepackage{tikz}
% Optional PGF libraries
\usepackage{pgflibraryarrows}
\usepackage{pgflibrarysnakes}
\usepackage{pgfplots}
\usepackage{fancybox}
\usepackage{listings}
\usepackage[abbr]{harvard}
\usepackage{hyperref}
\hypersetup{colorlinks=true,urlcolor=blue}
\usepackage[english]{babel}
% or whatever

\usepackage[latin1]{inputenc}
% or whatever

\usepackage{times}
\usepackage[T1]{fontenc}
% Or whatever. Note that the encoding and the font should match. If T1
% does not look nice, try deleting the line with the fontenc.


\usepackage{listings}

% "define" Scala
\lstdefinelanguage{scala}{
  morekeywords={abstract,case,catch,class,def,%
    do,else,extends,false,final,finally,%
    for,if,implicit,import,match,mixin,%
    new,null,object,override,package,%
    private,protected,requires,return,sealed,%
    super,this,throw,trait,true,try,%
    type,val,var,while,with,yield},
  otherkeywords={=>,<-,<\%,<:,>:,\#,@},
  sensitive=true,
  morecomment=[l]{//},
  morecomment=[n]{/*}{*/},
  morestring=[b]",
  morestring=[b]',
  morestring=[b]""",
}

\usepackage{color}
\definecolor{dkgreen}{rgb}{0,0.6,0}
\definecolor{gray}{rgb}{0.5,0.5,0.5}
\definecolor{mauve}{rgb}{0.58,0,0.82}

% Default settings for code listings
\lstset{frame=tb,
  language=scala,
  aboveskip=2mm,
  belowskip=2mm,
  showstringspaces=false,
  columns=flexible,
  basicstyle={\scriptsize\ttfamily},
  numbers=none,
  numberstyle=\tiny\color{gray},
  keywordstyle=\color{blue},
  commentstyle=\color{dkgreen},
  stringstyle=\color{mauve},
  frame=single,
  breaklines=true,
  breakatwhitespace=true,
  keepspaces=true
  %tabsize=3
}


% If you wish to uncover everything in a step-wise fashion, uncomment
% the following command:

% \beamerdefaultoverlayspecification{<+->}


\begin{document}

\begin{frame}
  \titlepage
\end{frame}


%------------------------------------------------------------------------
\begin{frame}[fragile]{The Collections Framework}

\begin{center}
\includegraphics[width=4in]{colls-coreInterfaces.png}
\end{center}

\begin{itemize}
\item A {\it collection} is an object that represents a group of objects.
\item The collections framework allows different kinds of collections to be dealt with in an implementation-independent manner.
\end{itemize}


\end{frame}
%------------------------------------------------------------------------

%------------------------------------------------------------------------
\begin{frame}[fragile]{Collection Framework Components}

The Java collections framework consists of:
\begin{itemize}
\item Collection interfaces representing different types of collections (sets, lists, etc)
\item General purpose implementations (like {\tt ArrayList} or {\tt HashSet})
\item Absract implementations to support custom implementations
\item Algorithms defined in static utility methods that operate on collections (like {\tt Collections.sort(List<T> list)})
\item {\bf Infrastructure interfaces that support collections (like {\tt Iterator})}
\end{itemize}
Today we'll learn a few basic concepts, then tour the collections library.
\end{frame}
%------------------------------------------------------------------------

%------------------------------------------------------------------------
\begin{frame}[fragile]{The {\tt Collection} Interface}


{\tt Collection} is the root interface of the collections framework, declaring basic operations such as:
\begin{itemize}
\item {\tt add(E e)} to add elements to the collection
\item {\tt contains(Object key)} to determine whether the collection contains {\tt key}
\item {\tt isEmpty()} to test the collection for emptiness
\item {\bf {\tt iterator()} to get an interator over the elements of the collection}
\item {\tt remove(Object o)} to remove a single instance of {\tt o} from the collection, if present
\item {\tt size()} to find out the number of elements in the collection
\end{itemize}
None of the collection implementations in the Java library implement {\tt Collection} directly.  Instead they implement {\tt List} or {\tt Set}.

\end{frame}
%------------------------------------------------------------------------


%------------------------------------------------------------------------
\begin{frame}[fragile]{Iterators}


Iterators are objects that provide access to the elements in a collection.  In Java iterators are represented by the {\tt Iterator} interface, which contains three methods:
\begin{itemize}
\item {\tt hasNext()} returns true if the iteration has more elements.
\item {\tt next()} returns the next element in the iteration.
\item {\tt remove()} removes from the underlying collection the last element returned by the iterator (optional operation).
\end{itemize}

The most basic and common use of an iterator is to traverse a collection (visit all the elements in a collection):
\begin{lstlisting}[language=Java]
ArrayList tasks = new ArrayList();
// ...
Iterator tasksIter = tasks.iterator();
while (tasksIter.hasNext()) {
    Object task = tasksIter.next();
    System.out.println(task);
}
\end{lstlisting}

See \link{\code/collections/ArrayListBasics.java}{ArrayListBasics.java} for examples.

\end{frame}
%------------------------------------------------------------------------

%------------------------------------------------------------------------
\begin{frame}[fragile]{The Iterable Interface}

An instance of a class that implements the {\tt Iterable} interface can be the target of a for-each loop.  The {\tt Iterable} interface has one abstract method, {\tt iterator}:

\begin{lstlisting}[language=Java]
public interface Iterable<T> {
    Iterator<T> iterator();
}
\end{lstlisting}

Let's see how we can implement an iterator for \link{\code/collections/DynamicArray.java}{DynamicArray.java}

\end{frame}
%------------------------------------------------------------------------

%------------------------------------------------------------------------
\begin{frame}[fragile]{DynamicArray}

\link{\code/collections/DynamicArray.java}{DynamicArray.java} is like an {\tt ArrayList}

\begin{lstlisting}[language=Java]
public class DynamicArray<E> implements Iterable<E> {

    private class DynamicArrayIterator implements Iterator<E> {
        ???
    }
    private Object[] elements;
    private int lastIndex;

    public DynamicArray() { this(10); }
    public DynamicArray(int capacity) { ... }
    public Iterator<E> iterator() { return new DynamicArrayIterator(); }
    public void add(E item) { ... }
    public E get(int index) { ... }
    public void set(int index, E item) { ... }
    public int size() { ... }
    public E remove(int index) { ... }
}
\end{lstlisting}

Assuming the methods above are defined, how do we write {\tt DynamicArrayIterator}?

\end{frame}
%------------------------------------------------------------------------

%------------------------------------------------------------------------
\begin{frame}[fragile]{DynamicArrayIterator}

The key component of an iterator is a {\it cursor}: a pointer to the next element in the collection.

\begin{itemize}
\item Since {\tt DynamicArray} uses an array as its backing data store, the cursor is simply an index into this array
\item The first element to be accessed is at index 0
\end{itemize}

\begin{lstlisting}[language=Java]
public class DynamicArray<E> implements Iterable<E> {
    private class DynamicArrayIterator implements Iterator<E> {
        private int cursor = 0;

        public boolean hasNext() {
            return cursor <= lastIndex;
        }
        public E next() {
            if (!hasNext()) { throw new NoSuchElementException(); }
            E answer = get(cursor++);
            return answer;
        }
        public void remove() {
            DynamicArray.this.remove(cursor - 1);
        }
    }
    private Object[] elements;
    private int lastIndex;
\end{lstlisting}

\end{frame}
%------------------------------------------------------------------------

%------------------------------------------------------------------------
\begin{frame}[fragile]{DynamicArrayIterator's {\tt next} Method}

An {\tt Iterator}'s {\tt next} method

\begin{itemize}
\item returns the element the cursor currently points to, and
\item moves the cursor to the next element in the collection
\end{itemize}

\begin{lstlisting}[language=Java]
public class DynamicArray<E> implements Iterable<E> {
    private class DynamicArrayIterator implements Iterator<E> {
        private int cursor = 0;

        public boolean hasNext() { ... }
        public E next() {
            if (!hasNext()) { throw new NoSuchElementException(); }
            E answer = get(cursor++);
            return answer;
        }
        public void remove() { ... }
    }
    private Object[] elements;
    private int lastIndex;
\end{lstlisting}

\end{frame}
%------------------------------------------------------------------------

%------------------------------------------------------------------------
\begin{frame}[fragile]{DynamicArrayIterator's {\tt hasNext} Method}

An {\tt Iterator}'s {\tt hasNext} method

\begin{itemize}
\item is used by clients of the {\tt Iterator} to determine whether unvisited elements of the collection remain
\item for {\tt DynamicArray} we simply test whether the cursor is still a valid array index
\end{itemize}

\begin{lstlisting}[language=Java]
public class DynamicArray<E> implements Iterable<E> {
    private class DynamicArrayIterator implements Iterator<E> {
        private int cursor = 0;

        public boolean hasNext() {
            return cursor <= lastIndex;
        }
        public E next() {
            if (!hasNext()) { throw new NoSuchElementException(); }
            E answer = get(cursor++);
            return answer;
        }
        public void remove() { ... }
    }
    private Object[] elements;
    private int lastIndex;
\end{lstlisting}

\end{frame}
%------------------------------------------------------------------------

%------------------------------------------------------------------------
\begin{frame}[fragile]{DynamicArrayIterator's {\tt remove} Method}
\vspace{-.05in}
\begin{itemize}
\item removes the last element returned by the iterator
\item the only safe way to modify a collection being iterated over
\end{itemize}
We simply use the {\tt DynamicArray}'s '{\tt remove} method
\vspace{-.05in}
\begin{lstlisting}[language=Java]
public class DynamicArray<E> implements Iterable<E> {
    private class DynamicArrayIterator implements Iterator<E> {
        private int cursor = 0;
        public boolean hasNext() { return cursor <= lastIndex; }
        public E next() {
            if (!hasNext()) { throw new NoSuchElementException(); }
            E answer = get(cursor++);
            return answer;
        }
        public void remove() {
            DynamicArray.this.remove(cursor - 1);
        }
    }
\end{lstlisting}
\vspace{-.05in}
Notice the syntax for distinguishing between the enclosing class's {\tt remove} method and the inner class's {\tt remove} method.\\
What if we called the inner class's {\tt remove} method recursively?
\end{frame}
%------------------------------------------------------------------------

%------------------------------------------------------------------------
\begin{frame}[fragile]{The Iterable Interface and teh For-Each Loop}

An instance of a class that implements {\tt Iterable} can be the target of a for-each loop.
\begin{lstlisting}[language=Java]
        DynamicArray<String> da = new DynamicArray<>(2);
        da.add("Stan");
        da.add("Kenny");
        da.add("Cartman");
        System.out.println("da contents:");
        for (String e: da) {
            System.out.println(e);
        }

\end{lstlisting}

See \link{\code/collections/DynamicArray.java}{DynamicArray.java} for implementation details.

\end{frame}
%------------------------------------------------------------------------

%------------------------------------------------------------------------
\begin{frame}[fragile]{Streams and Pipelines}

A stream is a sequence of elements.
\begin{itemize}
\item Unlike a collection, it is not a data structure that stores elements.
\item Unlike an iterator, streams do not allow modification of the underlying source
\end{itemize}

A stream carries values from a source through a pipeline.\\

\vspace{.1in}

A pipeline contains the following components:

\begin{itemize}
\item A source: This could be a collection, an array, a generator function, or an I/O channel.
\item Zero or more intermediate operations. An intermediate operation, such as filter, produces a new stream
\item A terminal operation. A terminal operation, such as forEach, produces a non-stream result, such as a primitive value (like a double value), a collection, or in the case of forEach, no value at all.
\end{itemize}

\end{frame}
%------------------------------------------------------------------------

%------------------------------------------------------------------------
\begin{frame}[fragile]{Stream Example: How Many Mustaches?}

Consider this simple example from \link{\code/collections/SuperTroopers.java}{SuperTroopers.java}:
\begin{lstlisting}[language=Java]
long mustaches = troopers.stream().filter(Trooper::hasMustache).count();
System.out.println("Mustaches: " + mustaches);
\end{lstlisting}

\begin{itemize}
\item {\tt troopers.stream()} is the {\it source}
\item {\tt .filter(Trooper::hasMustache)} is an {\it intermediate operation}
\item {\tt .count()} is the {\it terminal operation}
\end{itemize}
The terminal operation yields a new value which results from applying all the intermediate operations and finally the terminal operation to the source.

\end{frame}
%------------------------------------------------------------------------

%------------------------------------------------------------------------
\begin{frame}[fragile]{A Bigger Stream Example: WordCount Pipeline}

Consider this example from {\tt WordCount}s:
\begin{lstlisting}[language=Java]
Set<String> stopWords = new HashSet<>(Arrays.asList(
    "a", "an", "and", "are", "as", "be", "by", "is", "in", "of",
    "for", "from", "not", "to", "the", "that", "this", "with", "which"
));
wc.wordCounts.entrySet().stream()
    .filter(entry -> !stopWords.contains(entry.getKey().toLowerCase()))
    .sorted((e1, e2) -> e1.getValue() - e2.getValue())
    .forEach(entry ->
        System.out.printf("%s occurs %d times%n", entry.getKey(),
                                                  entry.getValue()));
\end{lstlisting}

This code does the same tasks we did before with classes and for loops.

\end{frame}
%------------------------------------------------------------------------

%------------------------------------------------------------------------
\begin{frame}[fragile]{WordCount Pipeline - Stop Words}

\begin{lstlisting}[language=Java]
Set<String> stopWords = new HashSet<>(Arrays.asList(
    "a", "an", "and", "are", "as", "be", "by", "is", "in", "of",
    "for", "from", "not", "to", "the", "that", "this", "with", "which"
));

\end{lstlisting}


\begin{itemize}
\item Every document has information-carrying words and grammatical words that carry no information, like prepositions, verbs like to be or have, pronouns
\item In document processing we call these non-information-carrying words {\it stop words}
\end{itemize}

Here we've implemented a naiive and terribly incomplete stop words list.\\

\vspace{.1in}

BTW, why a {\tt HashSet}?

\end{frame}
%------------------------------------------------------------------------

%------------------------------------------------------------------------
\begin{frame}[fragile]{WordCount Pipeline - filter}

Consider this example from {\tt WordCount}s:
\begin{lstlisting}[language=Java]
Set<String> stopWords = new HashSet<>(Arrays.asList(
    "a", "an", "and", "are", "as", "be", "by", "is", "in", "of",
    "for", "from", "not", "to", "the", "that", "this", "with", "which"
));
wc.wordCounts.entrySet().stream()
    .filter(entry -> !stopWords.contains(entry.getKey().toLowerCase()))
    .sorted((e1, e2) -> e1.getValue() - e2.getValue())
    .forEach(entry ->
        System.out.printf("%s occurs %d times%n", entry.getKey(),
                                                  entry.getValue()));
\end{lstlisting}

The filter operation takes a predicate function.
\begin{itemize}
\item A predicate function returns a {\tt boolean}
\item If predicate function returns {\tt true}, element is retained in the stream
\end{itemize}
Notice that we're also normalizing words to lower case.

\end{frame}
%------------------------------------------------------------------------

%------------------------------------------------------------------------
\begin{frame}[fragile]{WordCount Pipeline - sorted}

Consider this example from {\tt WordCount}s:
\begin{lstlisting}[language=Java]
Set<String> stopWords = new HashSet<>(Arrays.asList(
    "a", "an", "and", "are", "as", "be", "by", "is", "in", "of",
    "for", "from", "not", "to", "the", "that", "this", "with", "which"
));
wc.wordCounts.entrySet().stream()
    .filter(entry -> !stopWords.contains(entry.getKey().toLowerCase()))
    .sorted((e1, e2) -> e1.getValue() - e2.getValue())
    .forEach(entry ->
        System.out.printf("%s occurs %d times%n", entry.getKey(),
                                                  entry.getValue()));
\end{lstlisting}

The sorted operation takes a {\tt Comparator} that defines the ordering over the stream's elements.

\end{frame}
%------------------------------------------------------------------------

%------------------------------------------------------------------------
\begin{frame}[fragile]{WordCount Pipeline - forEach}

Consider this example from {\tt WordCount}s:
\begin{lstlisting}[language=Java]
Set<String> stopWords = new HashSet<>(Arrays.asList(
    "a", "an", "and", "are", "as", "be", "by", "is", "in", "of",
    "for", "from", "not", "to", "the", "that", "this", "with", "which"
));
wc.wordCounts.entrySet().stream()
    .filter(entry -> !stopWords.contains(entry.getKey().toLowerCase()))
    .sorted((e1, e2) -> e1.getValue() - e2.getValue())
    .forEach(entry ->
        System.out.printf("%s occurs %d times%n", entry.getKey(),
                                                  entry.getValue()));
\end{lstlisting}

{\tt forEach} is the terminal operation.
\begin{itemize}
\item Called for its effect - no return value
\end{itemize}

The underlying source is not modified.

\end{frame}
%------------------------------------------------------------------------


%------------------------------------------------------------------------
\begin{frame}[fragile]{Homework: Make the WordCount Pipeline Clearer}

Notice that we use anonymous lambda expressions in our WordCOunt pipeline:
\begin{lstlisting}[language=Java]
wc.wordCounts.entrySet().stream()
    .filter(entry -> !stopWords.contains(entry.getKey().toLowerCase()))
    .sorted((e1, e2) -> e1.getValue() - e2.getValue())
    .forEach(entry ->
        System.out.printf("%s occurs %d times%n", entry.getKey(),
                                                  entry.getValue()));
\end{lstlisting}

\begin{itemize}
\item Functional-style code can easily become hard to read.
\item You can improve readability by introducing intermediate helper variables with informative names.
\end{itemize}
Rewrite the WordCount pipeline with intermediate helper variables so that the pipeline is easy to understand.  You'll need to look up these aggregate operations in the Java API to get the types for these variables.

\end{frame}
%------------------------------------------------------------------------


% %------------------------------------------------------------------------
% \begin{frame}[fragile]{}


% \begin{lstlisting}[language=Java]

% \end{lstlisting}

% \begin{itemize}
% \item
% \end{itemize}


% \end{frame}
% %------------------------------------------------------------------------


\end{document}
